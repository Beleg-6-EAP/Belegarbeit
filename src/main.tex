%% The first command in your LaTeX source must be the \documentclass command.
\documentclass[acmtog]{acmart}
\usepackage[english,ngerman]{babel}
\usepackage[utf8]{inputenc} 

%% \BibTeX command to typeset BibTeX logo in the docs
\AtBeginDocument{%
  \providecommand\BibTeX{{%
    \normalfont B\kern-0.5em{\scshape i\kern-0.25em b}\kern-0.8em\TeX}}}
    
\copyrightyear{2024}
\acmYear{2024}
\citestyle{acmauthoryear}

\usepackage[figurename=Fig.]{caption}
\usepackage{csquotes}
\setcopyright{none}
\makeatletter
\renewcommand{\fnum@figure}{Abb. \thefigure}
\makeatother
\addto\captionsngerman{\renewcommand{\figurename}{Abb.}}
\settopmatter{printacmref=false} % Removes citation information below abstract
\renewcommand\footnotetextcopyrightpermission[1]{} % removes footnote with conference information in first column

%%
%% end of the preamble, start of the body of the document source.
\begin{document}

%%
%% The "title" command has an optional parameter,
%% allowing the author to define a "short title" to be used in page headers.
\title{Enterprise Architektur-Muster}

%%
%% The "author" command and its associated commands are used to define
%% the authors and their affiliations.
%% Of note is the shared affiliation of the first two authors, and the
%% "authornote" and "authornotemark" commands
%% used to denote shared contribution to the research.
\author{Julian Bruder}
\authornote{Alle Studierenden trugen zu gleichen Teilen zu dieser Arbeit bei.}
\author{Abdellah Filali}
\authornotemark[1]
\author{Luca Franke}
\authornotemark[1]
\affiliation{%
  \institution{Hochschule für Technik, Wirtschaft und Kultur Leipzig (HTWK Leipzig)}
  \streetaddress{Karl-Liebknecht-Str. 132}
  \city{Leipzig}
  %\state{Ohio}
  \country{Deutschland}
  \postcode{04277}
}
%%
%% By default, the full list of authors will be used in the page
%% headers. Often, this list is too long, and will overlap
%% other information printed in the page headers. This command allows
%% the author to define a more concise list
%% of authors' names for this purpose.
\renewcommand{\shortauthors}{Bruder, Filali, Franke}

%%
%% The abstract is a short summary of the work to be presented in the
%% article.
\begin{abstract}
Blah \ldots
\end{abstract}

\maketitle

\section{Einleitung}
% (Beschreibung von Kontext, Problemen, Anforderungen und Zielen)
Blah \ldots

\section{Grundlagen von Enterprise-Architekturen}
Blah \ldots

\section{Klassische Enterprise-Architekturen}
Blah \ldots

\section{Moderne Enterprise-Architekturen}

\subsection{Event-Driven Architecture}
Die Event-Driven Architecture wählt als Basis einen anderen Ausgangspunkt als die bisherigen Architekturmuster.
Während bei letzteren Komponenten Dienste bereitstellen, welche von anderen Komponenten explizit genutzt werden,
verhalten sich Dienst-bereitstellende Komponenten in der Event-Driven Architecture reaktiv,
werden also implizit von Dienst-konsumierenden Komponenten genutzt \cite{garlanShawImplizit}.
Ein System reagiert somit asynchron auf Zustandsänderungen, also Ereignisse in diesem System \cite{eda}.
Die in dieser Architektur minimalen Einheiten, welche Informationen einer Zustandsänderung kapseln, werden \textit{Events} genannt.
Die Idee der impliziten Behandlung von Ereignissen ist nicht neu und taucht erstmals 1994 im von Garlan und Shaw publizierten Papier
\textit{\enquote{An introduction to Software Architecture}} auf.

Betrachten wir im Folgenden die Basis-Bestandteile der Event-Driven Architecture:
\begin{itemize}
  \item Ereignis (englisch \textit{Event}): Kapselt Information einer Zustandsänderung eines Systems
  \item Produzent (englisch \textit{Producer}): Komponente, die Event erzeugt
  \item Herausgeber (englisch \textit{Publisher}): Komponente, die, von Produzenten erzeugte, Events publiziert
  \item Konsument (englisch \textit{Consumer}): Reagiert auf publizierte Events
  \item Vermittler (englisch \textit{Mediator}): Liegt zwischen Produzenten und Konsumenten - filtert Events und verteilt diese auf Konsumenten
  \item Event-Bus: Oft auch \textit{Event-Broker} genannt - bietet die Infrastruktur für die Gesamtheit der Vermittler
\end{itemize}
Abstrakt kann ein Event als ein Vertrag zwischen Produzenten und Konsumenten am Event-Bus betrachtet werden.
Der Konsument nutzt die Spezifikation des Events am Bus, der Produzent implementiert jene Spezifikation.
Abbildung \ref{fig:eda} stellt diesen Vertrag dar.

\begin{figure}[!h]
  \centering
  \includegraphics[width=\linewidth]{images/eda/eda.drawio}
  \caption{Vertrag zwischen Produzenten und Konsumenten am Event-Bus}
  \label{fig:eda}
\end{figure}

Durch den Vertrag weisen die Events am Event-Bus starke Kohäsion und somit lose Kopplung auf.
Diese lose Kopplung minimiert nicht nur kaskadierende Fehler, sondern ermöglicht agilen Entwickler-Teams durch klar abgegrenzte Features einfach definierbare Iterationen
- eine Menge von Events, deren Erzeugung und Konsumierung.

Weiter sind Events oft nah an dem, was Ereignisse in realen Prozessen sind, also domain-driven.
Gebündelt ermöglichen obige Punkte die kontinuierliche Auslieferung von Software in kurzen Intervallen.

Außerdem garantiert die asynchrone Behandlung von Ereignissen zusammen mit der loosen Kopplung maximale Skalierung.
Daher sind Event-Driven Architekturen besonders für datenintensive Echtzeit-Anwendungen wie IoT (Internet of Things) und Analytics geeignet \cite{iotEda}.

Die Agilität der Architektur kann weiter erhöht werden, indem der event-basierte Aspekt mit weiteren agilen Strukturen wie Microservices oder cloud-nativen Serverless-Functions kombiniert wird.
Die damit einhergehende Komplexität stellt teilweise hohe Anforderungen an die Entwickler.
Aufgrund der Asynchronität der Behandlung von Ereignissen ist die Testung des Systems meist schwer und die Fehlerbehandlung essentiell.
Mögliche Problemquellen schließen dabei unter anderem Event-Verlust, erhöhte Latenz und Inkonsistenz ein.
Die hohen Anforderungen an die Entwickler verlangen viel Vertrauen in jene, einer der zentralen Punkte des agilen Manifests \cite{agileManifesto}.
Insgesamt weist die Event-Driven Architecture also eine sehr hohe Agilität auf und ist damit besonders für moderne Software und ihre stetig wechselnden Anforderungen geeignet.

% TODO: Jetzt bestenfalls übergreifendes Beispiel

\section{Fallstudien und Praxisbeispiele}
Blah \ldots

\section{Diskussion}

\section{Zusammenfassung und Ausblick}
%(Überblick über die gesamte Arbeit, Rückführung auf Aussagen aus Kapitel 1 durchführen, offene Punkte als neue Forschungsfragen definieren)






\bibliographystyle{ACM-Reference-Format}
\bibliography{main}

\appendix

\section{Anhang 1}

\subsection{Übungsaufgaben}
Blah \ldots

\section{Anhang 2}
Blah \ldots

\end{document}
\endinput
