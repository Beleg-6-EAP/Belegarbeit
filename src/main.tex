%% The first command in your LaTeX source must be the \documentclass command.
\documentclass[acmtog]{acmart}
\usepackage[english,ngerman]{babel}
\usepackage[utf8]{inputenc} 

%% \BibTeX command to typeset BibTeX logo in the docs
\AtBeginDocument{%
  \providecommand\BibTeX{{%
    \normalfont B\kern-0.5em{\scshape i\kern-0.25em b}\kern-0.8em\TeX}}}
    
\copyrightyear{2024}
\acmYear{2024}
\citestyle{acmauthoryear}

\usepackage[figurename=Fig.]{caption}
\setcopyright{none}
\makeatletter
\renewcommand{\fnum@figure}{Abb. \thefigure}
\makeatother
\addto\captionsngerman{\renewcommand{\figurename}{Abb.}}
\settopmatter{printacmref=false} % Removes citation information below abstract
\renewcommand\footnotetextcopyrightpermission[1]{} % removes footnote with conference information in first column

%%
%% end of the preamble, start of the body of the document source.
\begin{document}

%%
%% The "title" command has an optional parameter,
%% allowing the author to define a "short title" to be used in page headers.
\title{Enterprise Architektur-Muster}

%%
%% The "author" command and its associated commands are used to define
%% the authors and their affiliations.
%% Of note is the shared affiliation of the first two authors, and the
%% "authornote" and "authornotemark" commands
%% used to denote shared contribution to the research.
\author{Julian Bruder}
\authornote{Alle Studierenden trugen zu gleichen Teilen zu dieser Arbeit bei.}
\author{Abdellah Filali}
\authornotemark[1]
\author{Luca Franke}
\authornotemark[1]
\affiliation{%
  \institution{Hochschule für Technik, Wirtschaft und Kultur Leipzig (HTWK Leipzig)}
  \streetaddress{Karl-Liebknecht-Str. 132}
  \city{Leipzig}
  %\state{Ohio}
  \country{Deutschland}
  \postcode{04277}
}
%%
%% By default, the full list of authors will be used in the page
%% headers. Often, this list is too long, and will overlap
%% other information printed in the page headers. This command allows
%% the author to define a more concise list
%% of authors' names for this purpose.
\renewcommand{\shortauthors}{Bruder, Filali, Franke}

%%
%% The abstract is a short summary of the work to be presented in the
%% article.
\begin{abstract}
Blah \ldots
\end{abstract}

\maketitle

\section{Einleitung}
% (Beschreibung von Kontext, Problemen, Anforderungen und Zielen)
Blah \ldots

\section{Grundlagen von Enterprise-Architekturen}
Blah \ldots

\section{Klassische Enterprise-Architekturen}
Blah \ldots

\section{Fallstudien und Praxisbeispiele}
Blah \ldots \cite{dummy}

\section{Diskussion}

\section{Zusammenfassung und Ausblick}
%(Überblick über die gesamte Arbeit, Rückführung auf Aussagen aus Kapitel 1 durchführen, offene Punkte als neue Forschungsfragen definieren)






\bibliographystyle{ACM-Reference-Format}
\bibliography{main}

\appendix

\section{Anhang 1}

\subsection{Übungsaufgaben}
Blah \ldots

\section{Anhang 2}
Blah \ldots

\end{document}
\endinput
